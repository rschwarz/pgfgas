%%
%% Copyright 2011, Robert Schwarz <schwarz@zib.de>
%%
%% PGF Gas
%%
%% This program is free software: you can redistribute it and/or modify
%% it under the terms of the GNU General Public License as published by
%% the Free Software Foundation, either version 3 of the License, or
%% (at your option) any later version.
%%
%% This program is distributed in the hope that it will be useful,
%% but WITHOUT ANY WARRANTY; without even the implied warranty of
%% MERCHANTABILITY or FITNESS FOR A PARTICULAR PURPOSE. See the
%% GNU General Public License for more details.
%%
%% You should have received a copy of the GNU General Public License
%% along with this program. If not, see <http:%%www.gnu.org/licenses/>.
%%

\documentclass[a4paper]{article}

\usepackage{listings}

\usepackage{tikz}
\usetikzlibrary{circuits}
\usepackage{pgfgas}

\begin{document}

\title{PGF Gas\\Using TikZ circuits to draw gas networks.}
\author{Robert Schwarz}
\date{\today}
\maketitle

This package defines four new symbols for common elements in natural
gas networks:

\begin{itemize}
\item[pi] Pipes
\item[cs] Compressors
\item[va] Valves
\item[cv] Control valves
\item[re] Resistors
\end{itemize}

Requires pgf with circuits library, e.g., version 2.10.

\begin{figure}
  \centering
  \begin{lstlisting}
    \begin{tikzpicture}[circuit]
      \draw (0,4) to [pi] (2,4);
      \draw (0,3) to [cs] (2,3);
      \draw (0,2) to [va] (2,2);
      \draw (0,1) to [cv] (2,1);
      \draw (0,0) to [re] (2,0);
    \end{tikzpicture}
  \end{lstlisting}

  \begin{tikzpicture}[circuit]
    \draw (0,4) to [pi] (2,4);
    \draw (0,3) to [cs] (2,3);
    \draw (0,2) to [va] (2,2);
    \draw (0,1) to [cv] (2,1);
    \draw (0,0) to [re] (2,0);
  \end{tikzpicture}
  \caption{All the standard elements are available.}
\end{figure}

\begin{figure}
  \centering
  \begin{lstlisting}
    \begin{tikzpicture}[circuit, scale=1.5, circuit symbol unit=10pt]
      \draw (0,0)    to [cv={fill=yellow}] (5,0);
      \draw (0,0)    to [cs={red}] (0,2);
      \draw (0,2)    to [va={fill=green}] (3,2);
      \draw (5,0)    to [re] (3,2);
      \draw (0,0)    to [cs={fill=red}] (3,2);
    \end{tikzpicture}
  \end{lstlisting}

  \begin{tikzpicture}[circuit, scale=1.5, circuit symbol unit=10pt]
    \draw (0,0)    to [cv={fill=yellow}] (5,0);
    \draw (0,0)    to [cs={red}] (0,2);
    \draw (0,2)    to [va={fill=green}] (3,2);
    \draw (5,0)    to [re] (3,2);
    \draw (0,0)    to [cs={fill=red}] (3,2);
  \end{tikzpicture}
  \caption{They can also be colored, filled, scaled and are automatically placed and rotated.}
\end{figure}

\end{document}